\documentclass[12pt,oneside]{article}
\input{prmb}

%\usepackage{kpfonts}
\usepackage{amsmath}

\makeatletter
\newcommand{\doublewidetilde}[1]{{%
  \mathpalette\double@widetilde{#1}%
}}
\newcommand{\double@widetilde}[2]{%
  \sbox\z@{$\m@th#1\widetilde{#2}$}%
  \ht\z@=.9\ht\z@
  \widetilde{\box\z@}%
}
\makeatother
\begin{document}
\title{\bf The Killing of a Sacred Idea}
\date{}
\author{F. Rosales}
\maketitle
The following is trivial and perhaps not all that rigorous. Regardless, I hope the argument is clear. 
\section{OLS}
Consider the usual model
\beq\label{ols}
\mY=\alpha_0\onevec+\alpha_1\xvec_1+\alpha_2\xvec_2+\alpha_3\xvec_3,\quad \xvec_3=\xvec_1\circ\xvec_2
\eeq
and its centred version
\beq\label{ols.centred}
\mY&=&\beta_0\onevec+\beta_1\tilde\xvec_1+\beta_2\tilde\xvec_2+\beta_3\tilde\xvec_3\nonumber\\
&=&\beta_0\onevec+\beta_1(\xvec_1-\bar\xvec_1)+\beta_2(\xvec_2-\bar\xvec_2)+\beta_3\{(\xvec_1\circ\xvec_2)-\overline{(\xvec_1\circ\xvec_2)}\}\nonumber\\
&=&\underbrace{\{\beta_0-\beta_1\bar\xvec_1-\beta_2\bar\xvec_2-\beta_3\overline{(\xvec_1\circ\xvec_2)}\}}_{\mbox{\small const.}}\onevec+\beta_1\xvec_1+\beta_2\xvec_2+\beta_3\xvec_3,
\eeq
which  has the same estimator as (\ref{ols}), except for the estimator for the intercept. 
Now consider the alternative model
\beq\label{ols.centredx}
\mY&=&\gamma_0\onevec+\gamma_1\tilde\xvec_1+\gamma_2\tilde\xvec_2+\gamma_3\doublewidetilde\xvec_3\nonumber\\
&=&\gamma_0\onevec+\gamma_1(\xvec_1-\bar\xvec_1)+\gamma_2(\xvec_2-\bar\xvec_2)+\gamma_3\{(\xvec_1-\bar\xvec_1)\circ(\xvec_2-\bar\xvec_2)\}\nonumber\\
&=&\underbrace{(\gamma_0-\gamma_1\bar\xvec_1-\gamma_2\bar\xvec_2+\gamma_3\bar\xvec_1\bar\xvec_2)}_{\mbox{\small const.}}\onevec+
\underbrace{(\gamma_1-\bar\xvec_2)}_{\mbox{\small const.}}\xvec_1+\underbrace{(\gamma_2-\bar\xvec_1)}_{\mbox{\small const.}}\xvec_2+\gamma_3\xvec_1\circ\xvec_2.
\eeq
Hence, Model (\ref{ols}), or any of its parametrizations, are solved via
\beq\label{ols.sol}
\hat\deltavec&=&\mbox{argmin}_{\deltavec}\left\{\evec(\deltavec)^\top\evec(\deltavec)\right\}\nonumber\\
\evec(\deltavec)&:=&\mY-\delta_0\onevec-\delta_1\xvec_1-\delta_2\xvec_2-\delta_3(\xvec_1\circ\xvec_2),
\eeq
for suitable definitions of $\deltavec$. Since (\ref{ols.sol}) has the same form  for models (\ref{ols})-(\ref{ols.centredx}), the solver leads to the
 same solution, so $\hat\mY$ must be the same in all cases. Regarding the parameters:
 \begin{enumerate}
 \item (\ref{ols}) and (\ref{ols.centred}) must have exactly the same estimators for additive and interaction effects, but the intercept should be different. %This is observed in the paper. 
 \item (\ref{ols}) and (\ref{ols.centredx}) must have different estimators for the additive effects and the intercept, but the coefficient related to interactions should be the same. %This is also observed in the paper. 
 \end{enumerate}

%%%%%%%%%%%%%%%%%%%%%%%%%%%%%%%%%%%%%%%%%%
\section{RRBLUP}
We can regularize the estimated coefficients via penalization by 
\beq
\hat\alphavec&=&\mbox{argmin}_{\alphavec}\{\evec(\alphavec)^\top\evec(\alphavec)+
\lambda(\alpha_1^2+\alpha_2^2+\alpha_3^2)\},\nonumber\\
\evec(\alphavec)&:=&
\mY-\alpha_0\onevec-\alpha_1\xvec_1-\alpha_2\xvec_2-\alpha_3(\xvec_1\circ\xvec_2)\label{rrblup1}
\eeq
%or by
%\beq
%\hat\gammavec&=&\mbox{argmin}_{\gammavec}\{\evec(\gammavec)^\top\evec(\gammavec)+
%\lambda(\gamma_1^2+\gamma_2^2+\gamma_3^2)\},\nonumber\\
%\evec(\gammavec)&:=&
%\mY-{(\gamma_0-\gamma_1\bar\xvec_1-\gamma_2\bar\xvec_2+\gamma_3\bar\xvec_1\bar\xvec_2)}\onevec-
%{(\gamma_1-\bar\xvec_2)}\xvec_1-\nonumber\\
%&&{(\gamma_2-\bar\xvec_1)}\xvec_2-\gamma_3(\xvec_1\circ\xvec_2)\label{rrblup1x}
%\eeq

This RRBLUP problems is  just the  OLS problem with a penalty on all terms except for the intercept. Moreover, if one assumes that the penalty is fixed to $\lambda=1$, the behaviour of the estimated coefficients  can be anticipated. Namely, 
\begin{enumerate}
\item If one extends solver  (\ref{ols.sol}) to (\ref{rrblup1}). The expected result is all shrinked estimators except for the intercept. In the paper the values $1.81$ and $1.83$ are reported for uncentred model (\ref{ols}) and $0.334$ and $0.334$ are reported for alternative  model (\ref{ols.centredx}).
\item If one applies solver (\ref{rrblup1}) to either model (\ref{ols}) or to (\ref{rrblup1}), the estimated parameters should all be different, but   $\hat\mY$ must be the same. {\color{red} This does not happen}.
\end{enumerate}  

If instead the alternative models are considered
\beq
\hat\alphavec&=&\mbox{argmin}_{\alphavec}\{\evec(\alphavec)^\top\evec(\alphavec)+
\lambda\alpha_3^2\},\nonumber\\
\evec(\alphavec)&:=&
\mY-\alpha_0\onevec-\alpha_1\xvec_1-\alpha_2\xvec_2-\alpha_3(\xvec_1\circ\xvec_2)\label{rrblup2}
\eeq
or
\beq
\hat\gammavec&=&\mbox{argmin}_{\gammavec}\{\evec(\gammavec)^\top\evec(\gammavec)+
\lambda\gamma_3^2\},\nonumber\\
\evec(\gammavec)&:=&\mY-{(\gamma_0-\gamma_1\bar\xvec_1-\gamma_2\bar\xvec_2+\gamma_3\bar\xvec_1\bar\xvec_2)}\onevec-
{(\gamma_1-\bar\xvec_2)}\xvec_1-\nonumber\\
&&{(\gamma_2-\bar\xvec_1)}\xvec_2-\gamma_3(\xvec_1\circ\xvec_2)\label{rrblup2x},
\eeq
as done in the last part of example 1, we observe some other things:
\begin{enumerate}
\item (\ref{rrblup1}) to (\ref{rrblup2}). The expect result is that the estimators $\hat\gamma_0$ and $\hat\gamma_3$ should remain the same, while all the others  must change. {\color{red} This does not hold empirically. Check once again}.
\item (\ref{rrblup1x}) to (\ref{rrblup2x}). The expect result is that the estimators $\hat\gamma_0$ and $\hat\gamma_3$ should remain the same, while all the others  must change. In the paper we have $0.334$ for the intercept in both models and $-0.575$ and $-0.570$ for models (\ref{rrblup1x}) and (\ref{rrblup2x}) respectively.
\item (\ref{rrblup2}) to (\ref{rrblup2x}). The expect result is that both models produce the same estimator for $\hat\gamma_3$ but all the others should be different,  which is verified. 
\end{enumerate}

\end{document}



